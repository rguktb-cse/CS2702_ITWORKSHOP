% Creating a simple Title Page in Beamer
\documentclass{beamer}

% Theme choice:
\usetheme{AnnArbor}
% \usetheme{CambridgeUS}
% \usetheme{Antibes}

% Title page details: 
\title{Your First \LaTeX{} Presentation}
% \subtitle{My-subtitle}
\author{latex-beamer.com}
% \institute{Online Beamer Tutorials}

% Multiple authors
% \author{First~Author \and 
%   Second~Author \and
%   Third~Author \and
%   Fourth~Author \and
%   Fifth~Author}

% Two authors with different affiliations
% \author{First Author\inst{1} \and Second Author\inst{2}}
% \institute{\inst{1} Affiliation of the 1st author \and
%  \inst{2} Affiliation of the 2nd author}

\date{\today}

% \logo{
% \includegraphics[width=3cm]{logo.jpg}
% }

% Logo only on title page
% \titlegraphic{
%     \includegraphics[width=2cm]{logo.jpg}
% }

% Modify footer text: 
% \title[Center text]{Your First \LaTeX{} Presentation}
% \subtitle{My-subtitle}
% \author[Left text]{latex-beamer.com}
% \date[Right text]{\today}

% Define a counter
% \newcounter{currentenumi}

\begin{document}

% Title page frame
\begin{frame}
    \titlepage
\end{frame}

% % Outline frame
% \begin{frame}{Outline}
%     \tableofcontents
% \end{frame}

% % % Presentation structure
% % \section{Problem statement}
% % \section{Existing results}
% %     \subsection{Method 1}
% %     \subsection{Method 2}
% %     \subsection{Method 3}
% % \section{Comparative study}
% % \section*{References}

% \begin{frame}
% % to enforce entries in the table of contents
% \end{frame}

% % Abstract environment
% \begin{abstract}
%   content
% \end{abstract}

% \begin{frame}{Ordered Lists in Beamer}
% \begin{enumerate}
%     \item Item 1
%     \item Item 2
%     \item Item 3
% \end{enumerate}
% \end{frame}

% \begin{frame}{Lists in multiple frames}{Frame 1}
% \begin{enumerate}
%     \item Item 1
%     \item Item 2
%     \item Item 3
% % Store the actual item number
%     \setcounter{currentenumi}{\theenumi}
% \end{enumerate}
% \end{frame}
% \begin{frame}{Lists in multiple frames}{Frame 2}
% \begin{enumerate}
% % Use the previous stored item number
% \setcounter{enumi}{\thecurrentenumi}
%     \item Item 4
%     \item Item 5
% \end{enumerate}
% \end{frame}

% %Columns
% \begin{frame}{Text and Image in beamer}
%     \begin{columns}
%     \column{0.4\textwidth}
%         This is an example of text and image in the same slide using columns environment.
%     \column{0.6\textwidth}
%         \begin{figure}
%         \centering
%         \includegraphics[width=\textwidth]{nn.png}
%         \caption{Neural Network with 5 neurons in the hidden layer. }
%         \end{figure}
%     \end{columns}
% \end{frame}



% \begin{frame}{Block environment}{Madrid theme}
% \begin{block}{Block title}
%     It can be useful to treat some content differently by putting it into a block. This can be done by using blocks!
% \end{block}
% \end{frame}

% \begin{frame}{Basic Blocks}
%     \begin{block}{Standard Block}
%         This is a standard block.
%     \end{block}
%     \begin{alertblock}{Alert Message}
%         This block presents alert message.
%     \end{alertblock}
%     \begin{exampleblock}{An example of typesetting tool}
%         Example: MS Word, \LaTeX{}
%     \end{exampleblock}
% \end{frame}


% \begin{frame}{My first table}
% \begin{tabular}{|c||l||r|}
% \hline
%     centered & left-aligned & right-aligned \\ 
% \hline
%     A & C & E\\ 
% \hline
%     B & D & F\\ 
% \hline
% \end{tabular}
% \end{frame}



\end{document}



